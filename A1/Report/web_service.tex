\chapter{Web Service}
\label{web_service}

The present section is reserved to the description of the \emph{Web Service} implemented by our group. We start this chapter with a brief description of the technology used in the development of the service, deepening into the \emph{Web Method} implemented.

\section{Technology}

As requested in the work's assignment, the \emph{Web Service} was developed with the \emph{Java API for XML Web Services (JAX-WS)}. \emph{JAX-WS} is a very popular and useful technology for building \emph{Web Services} (and \emph{Web Clients}) using \emph{XML} for their communications.

Like many other \emph{Java EE APIs}, \emph{JAX-WS} makes use of the power of \emph{Java Annotations} to enable a quick, simple and easy development of \emph{endpoints} and \emph{Web Services}. Thanks to the annotations used, the complexity of many operations is hidden from the programmers, which eases and speeds up the development process.

In \emph{JAX-WS} a \emph{Web Service} consists in one class with a finite set of (public) \emph{Web Methods}, whose invocations by a client are accomplished using \emph{SOAP}, a\emph{XML-based protocol} that defines the envelope structure, encoding rules, and conventions for representing web service invocations and their responses, transmitted over the \emph{HTTP} protocol. Once the \emph{Web Service} is fully developed it should be hosted in a given \emph{URL}, along with a description of the services created and available, so that client applications know what services to invoke and how to invoke them. This description, called \emph{Web Services Description Language (WSDL)}, is generated and managed by the mentioned \emph{API}, and consists in a language written in \emph{XML} describing the service.

The development of the \emph{Client Application} is also very simple and straightforward: A client creates a local object that represents the service to invoke, called a \emph{Proxy}, and invokes all the desired methods\footnote{Also called \emph{Services}} through that same \emph{Proxy}. The \emph{JAX-WS API} offers a routine for generating the \emph{Proxy} based on the \emph{Web Services Description Language (WSDL)}, which means that, in the client side, the programmer does not need to generate or program in any way the \emph{Proxy}.

Besides providing a very simple, straightforward, quick and effortless way of developing \emph{Web Servers} and \emph{Clients} the \emph{JAX-WS API} also offers the platform independence of the \emph{Java} programming language, without making it a design constraint: A \emph{Non-Java Client} can access a \emph{Web Service} developed in \emph{C++} or any other programming language and vice versa.

\section{Web Methods}

    This section is dedicated to the description of each method available in the web-service. This methods were implemented based on the specifications indicated in the wording of the assignment. Once all the parameters are integers, the first step consisted in convert them to \emph{doubles} avoiding some possible \emph{rounding} errors. However, the result is rounded and converted to \emph{integer}.

\subsection{mealtimeInsulinDose}

    The aim of this method is to calculate the number of insulin units needed after a given meal.

    This method takes the amount of carbohydrate in a given meal, and returns
the number of units of insulin needed after that meal. The returned
number of units of insulin equals the carbohydrate dose plus the high
blood sugar dose, which are computed as follows.
    
    The carbohydrate dose equals the total grams of carbohydrates in the meal
divided by the amount of carbohydrate disposed by one unit of insulin,
corrected by taking into account the personal sensitivity to insulin.
This dose equals to:

$$carbohydrateAmount / carbohydrateToInsulinRatio / personalSensitivity \times 50$$

    The high blood sugar dose equals the difference between the pre-meal
blood sugar level and the target blood sugar level, divided by the
personal sensitivity to insulin. This equals $(preMealBloodSugar -
targetBloodSugar) / personalSensitivity$. The personal sensitivity
is can be explicitly indicated by the user or calculated using the $personalSensitivityToInsulin()$ method.

    In the special case when the target blood sugar level is greater than the
pre-meal blood sugar level, the return value of this method is zero (no
insulin).

\subsection{backgroundInsulinDose}

    This method calculates the total number of units of insulin needed between meals.
    
    The total insulin units required in one day equals $0.55 \times body
weight$ in kilograms. This method returns 50\% of that number, since
the background need for insulin, between meals, is around half of the
daily total.

\subsection{personalSensitivityToInsulin}

    This method is used to calculate the sensitivity to insulin based on the psychical activity level and blood sugar drop.
    
    One unit of insulin typically drops blood sugar by 50 mg/dl, but this
value depends on each individual's sensitivity and daily physical
activity. This method predicts the blood sugar drop (in mg/dl) that will
result from one unit of insulin, for a given physical activity level.

    To predict the blood sugar drop, this method accepts two arrays with
K samples of (physical activity level, blood sugar drop). The two arrays
must therefore have the same length. First, a simple linear regression
(least squares method) is performed to compute alpha and beta. Then, the
return value is $alpha + beta \times physicalActivityLevel$.

    The physical activity levels, including the ones in the array of samples,
and the blood sugar drop values are non-negative integers. The return
value of this method may be passed to $mealtimeInsulinDose()$
as a parameter.

\section{Deployment}

    The developed \emph{Web Service} was hosted in a virtual machine located at the Department of Informatics Engineering, under the url \url{liis-lab.dei.uc.pt}. The service was hosted in port $8080$, and the \emph{WSDL} can be accessed via \url{liis-lab.dei.uc.pt:8080/Server?wsdl}
    
    In fact, this virtual machine does not correspond to the set of virtual machines requested by the students of this course. This machine was being used by \emph{Joaquim Leitão} in a research project supervised by \emph{Prof. Alberto Cardoso}.
    
    Since it was not being used at the time this assignment was developed, we asked \emph{Prof. Alberto Cardoso} permission to use this machine, thus avoiding the unnecessary creation of another virtual machine. We would also like to take this opportunity to thank \emph{Prof. Alberto Cardoso} for his cooperation.