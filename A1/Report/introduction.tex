\chapter{Introduction}

This document addresses the work developed by \emph{João Miguel Simões}, \emph{Joaquim Leitão} and \emph{Leonardo Toledo} as part of the first assignment of the \emph{Master's Degree in Informatics Engineering} \emph{Software Quality and Dependability} course.

In the present work, the development of a fully functional \emph{Web Service} based on \emph{N-Version} programming \emph{Insulin Dose Calculator} for patients with type-2 diabetes is proposed, as well as the development of a client interface for this system, aiming to ease its usage in an application scenario.

As studied and documented by medical specialists, type-2 diabetes patients need periodic insulin injections to help process blood sugar, thus, is it critical for such patients to obtain insulin in the right amount that allows them to metabolise the carbohydrates obtained from food.\footnote{It is even more critical to provide the right amount of insulin to the patient, since higher insulin administration doses may cause \emph{hypoglycemia}, possibly leading to coma and brain damage}.

Specialists have recommended that such patients seek to maintain a target glucose level somewhere between $80$ mg/dl and $120$ mg/dl throughout the day. To keep their glucose levels within the recommended values, type-2 diabetes patients regularly measure their blood glucose level and calculate the necessary insulin dose needed to inject in their bodies to process the excess blood sugar.

These patients inject regularly a certain number of \emph{insulin units} throughout the day, adding up to a \emph{daily insulin total number of units}. Part of this total is intended to cover the background insulin needed between meals, and is called \emph{basal insulin replacement}. The remaining, called \emph{bolus insulin replacement} is needed after every meal to process the carbohydrates obtained from food.

The main goal of the \emph{Insulin Dose Calculator} proposed in this assignment is to calculate and indicate the patient the insulin dose needed to take at different moments of the day. The patient will then input a number of parameters requested by the \emph{Insulin Dose Calculator} and it will compute the adequate number of insulin units the patient must take. Further more, the	system shall present the desired insulin dose within less than \emph{4 seconds} after the request was initially made by the user.

Given the fact that we are making use of \emph{N-Version} programming, upon any given request our system will contact a series of \emph{Web Services}\footnote{We used three \emph{Web Services} in our scenario} which will compute the adequate number of insulin units the patient must take in the given scenario. Since we now have many (possibly different) answers for the user's request, we need to, in a way, filter these values, selecting only the most adequate to present to the user. To accomplish this task we also developed a \emph{Voter System}, responsible for \emph{voting} the different answers, selecting our system's final (and assumed to be corrected) answer.

The remaining of this document is structured as follows: In chapter \ref{overview} we will perform a brief overview of the architecture of the application developed, detailing about its main components and interfaces. In chapter \ref{web_service} the creation and development of the \emph{Web Service} is addressed as well as the services it provides. In chapter \ref{interface_application} the client interface developed will be presented and discussed. In chapter \ref{voter} we will discuss the implementation of the voter system, presenting its overall operation mode and main implementation decisions made during its development. In this chapter we will also perform a formal verification of the voter, responsible for ensuring its correct behaviour. Finally, chapter \ref{conclusion} is reserved for the conclusion of our work, where we briefly reflect on the work done.